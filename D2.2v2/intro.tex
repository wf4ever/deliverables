\section{Introduction}

This deliverable describes Phase II of the design, implementation and
deployment of the Wf4Ever components that will support workflow
lifecycle management. The document should be read in tandem with other
Month 32 deliverables, in particular D3.2v2 (Design, implementation
and deployment of Workflow Evolution, Sharing and Collaboration
components -- Phase II)~\cite{D3.2v2} and D4.2v2 (Design,
implementation and deployment of Workflow Integrity and Authenticity
Maintenance components -- Phase II)~\cite{D4.2v2} which address
complementary aspects of the overall wf4ver architecture and
components. This deliverable supersedes D2.2v1, and can be read as a standalone document. 

According to the Description of Work, \emph{This prototype will include the following functionalities: new versions of the Research Object model and ontology network, advanced management functions (filtering, clustering, etc.), playback functionalities for reproducibility, and workflow classification, indexing and explanation techniques.}. 

These requirements are addressed in the following way:

Section~\ref{sec:romodel} presents the Research Object Model defined within Wf4Ever. Specifically, we present a family of ontologies for specifying Research Objects and their associated resources, e.g., workflow, workflow runs, etc. 

Section~\ref{sec:romt} presents the tools that we have developed for assisting users in creating and managing Research Objects. More specifically, Section~\ref{sec:romanager} presents the Research Object Manager (RO Manager), a command line tool for creating, displaying and manipulating Research Objects. Section~\ref{sec:myexperiment} shows how the myExperiment virtual environment \cite{DBLP:journals/fgcs/RoureGS09}, was extended to allow end-users, who are not necessarily information technology experts, to create, share, publish and curate Research Objects.

Section~\ref{sec:rodl} presents the Research Object Digital Library (RODL), which acts as a full-fledged back-end not only for scientists but also for librarians. 

Section~\ref{sec:abstract-indexing} presents the motif ontology that we developed for abstracting scientific workflows, and illustrates how it has been used to document workflows. It then goes on to present a solution that we developed for indexing workflows based on the processes (steps) they are composed of, with the purpose of assisting users in discovering workflows that are of interest to them. 
