\section{Summary}
\label{sec:conclusions}

We have presented in this deliverable the final version Research Object model defined within Wf4Ever, as a family of ontologies. We also presented the tools that were built on the model in order to facilitate the creation, curation and sharing of Research Objects, namely, the Research Object Manager (RO Manager), a command line tool for creating, displaying and manipulating Research Objects, RODL, which acts as a back-end, with two storage alternatives: a digital repository to keep the content, as a triple store to manage the metadata content, and the myExperiment virtual research environment, which was extended to allow end-users to create, upload, share and curate Research Objects. We also presented two models that cater for advanced functionalities, namely abstracting and indexing workflows.

Our ongoing and future work aims to advertise and disseminate the Research Object model and the tools developed around it. In this respect, it is worth mentioning that we have launched a website dedicated to Research Objects\footnote{\url{http://www.researchobject.org/}}, with examples that assist prospective adopters in understanding the model usage and benefits.