\section{The Research Object Model}
\label{sec:romodel}

The RO model consists of  a family of ontologies organized into core and extensions, which we will present in this section.

\subsection{RO core ontology} 
The Core RO Ontology provides the minimum terms that are essential to the specification of research objects. Specifically, it caters for two essential requirements by providing a container structure that can be used by the scientists to bundle the resources and material relevant for their investigation, and by enabling annotations of such a container, its resources, as well as the relationships between resources thereby making the research object interpretable and reusable. 

To cater for the specification of aggregation structures, we built the Research Object Core Ontology upon the popular ORE vocabulary. ORE defines standards for the description and exchange of aggregations of Web resources. 
Figure \ref{fig:ro_ontology} illustrates the main terms that constitute the Research Object Core Ontology, which we describe in what follows.


\begin{figure}[ht]
  \centering
  \includegraphics[width=0.6\textwidth]{Figures/ro_ontology_1.png}
  \caption{RO as an ORE aggregation.}
  \label{fig:ro_ontology}
\end{figure}

\begin{itemize}
\item
\texttt{ro:ResearchObject}\footnote{The namespace of the Research Object Core Ontology \texttt{ro} is \url{http://purl.org/net/wf4ever/ro\#}}, represents an aggregation of resources. It is a sub-class of \texttt{ore:Aggregation} and acts as an entry point to the research object.
\item
\texttt{ro:Resource}, represents a resource that can be aggregated within a research object and is a sub-class of \texttt{ore:AggregatedResource}. A resource can be a Dataset, Paper, Software or Annotation. Typically, a \texttt{ro:ResearchObject} aggregates multiple \texttt{ro:Resource}, and this relationship is specified using the property \texttt{ore:aggregates}.
\item
\texttt{ro:Manifest}, a sub-class of \texttt{ore:ResourceMap}, represents a resource that is used to describe a \texttt{ro:ResearchObject}. It plays a similar role to the manifest in a JAR or a ZIP file, and is primarily used to list the resources that are aggregated within the research object.
\end{itemize}

The second core requirement that, the Research Object Core Ontology caters for, is the descriptions of the research object and its elements. We chose the Annotation Ontology (AO) release 2.0b2~\cite{COG11}.To annotate research objects, we make use of the following three Annotation Ontology terms \texttt{ao:Annotation}\footnote{The namespace of \texttt{ao} is \url{http://purl.org/ao/}}, which represents the annotation itself; \texttt{ao:Target}, which is used to specify the \texttt{ro:Resource}(s) or \texttt{ro:ResearchObject}(s) subject to annotation; and \texttt{ao:Body}, which comprises a description of the target.
In the case of research objects, we use annotations as a mean for decorating a resource (or a set of resources) with metadata information. The body is specified in the form of a set of RDF statements, which can be used to, e.g., specify  the date of creation of the target or its relationship with other resources or research objects. Also, annotations can be provided for human consumption (e.g. a description of a hypothesis that is tested by a workflow-based experiment), or for machine consumption (e.g. a structured description of the provenance of results generated by a workflow run). Both kinds of annotations are accommodated using Annotation Ontology structures.

\subsection{RO Extension Ontologies}
We present in this section two  extensions to the core Research Object ontology. The first specializes the kinds of resources that the research object can aggregate. In particular, we present extensions to specify  method and experiments and the traces of their executions. The second kind of extension shows how specific metadata information, specifying the evolution of the research object over time, can be specified by specializing the Research Object core ontology.

\paragraph{Specifying Workflows}
To describe workflow research objects the workflow description vocabulary \textit{wfdesc}\footnote{The name space of \textit{wfdesc} is \url{http://purl.org/wf4ever/wfdesc\#}.} defines several specific resources that are involved in a workflow specification. The choice of these resources was performed by examining the commonalities between major data driven workflows, namely Taverna\footnote{http://www.taverna.org.uk}, Wings\footnote{http://http://wings-workflows.org} and Galaxy\footnote{http://galaxyproject.org}, to cite a few.

\begin{figure}[ht]
  \centering
  \includegraphics[width=0.6\textwidth]{Figures/wfdesc.png}
  \caption{The \textit{wfdesc} ontology.}
  \label{fig:wfdesc}
\end{figure}

Figure \ref{fig:wfdesc} illustrates the terms that compose the \textit{wfdesc} ontology. Using such ontology, a workflow is described using the following three main terms:
\begin{itemize}
\item
\texttt{wfdesc:Workflow} refers to a network in which the nodes are processes and the edges represent data links. It is defined as a subclass of the \textit{Plan} concept from the PROV-O ontology, which represents a set of actions or steps intended by one or more agents to achieve some goals \cite{w3c-prov-o}. 
\item
\texttt{wfdesc:Process} is used to describe a class of actions that when enacted give rise to process runs. Processes specify the software component (e.g., web service) responsible for undertaking those actions.
\item
\texttt{wfdesc:DataLink} is used to encode the data dependencies between the processes that constitute a workflow. Specifically, a data link connects the output of a given process to the input of another process, specifying that the artifacts produced by the former are used to feed the latter.
\end{itemize}


\paragraph{Describing Experimental Provenance using the \textit{wfprov} Vocabulary}
The \textit{wfprov} ontology is used to describe the provenance traces obtained by enacting  workflows. It is defined as an extension to the ongoing W3C PROV standard ontology - PROV-O\footnote{Note that the \textit{wfprov} is reported in the W3C PROV Working Group implementation report.}.

\begin{figure}[ht]
  \centering
  \includegraphics[width=0.6\textwidth]{Figures/wfprov.png}
  \caption{The \textit{wfprov} ontology.}
  \label{fig:wfprov}
\end{figure}

Figure \ref{fig:wfprov} illustrates the structure of the \textit{wfprov} ontology and its alignments with the W3C PROV-O ontology. A a workflow run~(\texttt{wfprov:WorkflowRun}) represents the enactment of a given workflow. It is composed of a set of process runs~(\texttt{wfprov:ProcessRun}), each representing the enactment of a process. A process run may use some artifacts~(\texttt{wfprov:Artifact}) as input and generate others as output. A process run is enacted by a workflow engine~(\texttt{wfprov:WorkflowEngine}), which can be seen as a PROV software agent.

By chaining the usage and generation of artifact together, the \textit{wfprov} ontology allows scientists to trace the lineage of workflow results. For example the user can identify the input artifacts that were used to feed the wokflow run (as a whole) to obtain a given output that was generated by the workflow run.

\paragraph{Tracking Research Object Evolution using the \textit{roevo} Vocabulary}
The \textit{roevo} ontology is another extension to the minimal core ontology for describing an important aspect of research objects, its life cycle.
%There are a number of existing work captures changes of information objects, like the changeset vocabulary, evolution of ontologies, like xxx, yyy, etc. 
To track the life cycle of a research object, we need to describe its changes at different levels of granularity, about the research object as a whole and about the individual resources. Also, we want to provide sufficient details to track the changes in order to roll back to a particular version or to quality control changes. Therefore, we need to describe when the change took place, who performed the change, and dependency relationships between the changes. %None of the existing vocabularies provide all the structure for us to build upon. 
Change is closely related to the provenance of a particular version of a research object or a resource. A study of the latest PROV-O ontology shows that it indeed provides all the foundational information elements for us to build the evolution ontology. 

\begin{figure}[ht]
  \centering
  \includegraphics[width=0.6\textwidth]{Figures/roevo.png}
  \caption{The \textit{roevo} ontology extending PROV-O core terms.}
  \label{fig:ro_evo}
\end{figure}    

Figure~\ref{fig:ro_evo} illustrates the core concepts of this ontology and how it extends the PROV-O:
\begin{itemize}
\item To capture different status of a research object we create three sub-classes of \texttt{ro:ResearchObject}: the \texttt{roevo:LiveRO} is a research object to capture research findings during a live investigation and it can changed, and it can either archived or snapshotted. The \texttt{roevo:ArchivedRO} can be regarded as a production research object to be preserved and archived, such as one describing findings published in an article, and it can no longer be changed; the \texttt{roevo:SnapshotRO} represents a live Research Object at a particular time.
\item Both a snapshot of a live Research Object and an archived Research Object can be regarded as a versioned Research Object, i.e. a \texttt{roevo:VersionableResource}, Because it is a sub-class of \texttt{prov:Entity}, we can reuse PROV-O properties to describe the provenance or changes of this entity, such as pointing to the activity leading to any of its changes, the source research object that it was derived from, and the agent involved in its change.
\item A change is a \texttt{prov:Activity}, which means that it has a start time, an end time, an input entity and a resulting entity. Also a change leading to a new Research Object can constitute a series of changes. Therefore, we have a composite \texttt{roevo:ChangeSpecification} activity, which has a number of unit \texttt{roevo:Change}s. A unit change can be adding, removing or modifying a resource or a research object. But these different changes share the same pattern of taking an input entity and producing an output entity, which can all be nicely covered by properties from PROV-O.
\end{itemize}
