\section{Research Object-Enabled myExperiment}
\label{sec:myexperiment}

%This section describes the efforts that went into incorporating research objects within myExperiment. In particular, how the notion of myExperiment pack was used as a starting point to incorporate new features/functionalaities. We will also discuss the diferent iterations that involved Wf4ever and Biovel users in those developments.

In this section, we describe how myExperiment \cite{} was extended in order to cater for the sharing, publication and curation of research objects. myExperiment is a virtual research environment targeted towards collaborations for sharing and publishing workflows (and experiments). myExperiment provides mechanisms for the functionalities necessary for sharing workflows within and across multiple communities. In doing so, myExperiment adopts a social web approach, which is adapted to the need of scientist. The workflows that are shared using myExperiment do not need to be specified in a particular workflow management system. For example, we find on myExpeirment workflows that have been specified using Galaxy \cite{}, Taverna \cite{}, Kepler \cite{} and Vistrails \cite{}.

While initially targeted towards workflows, the creators of myExperiment were aware that scientists wants to share more than just workflows and experiments. Because of this, myExperiemnt was extended to supports the sharing of artifacts known as Packs. A pack can be seen as a basic aggregation of resources, which can be workflows, but also files, presentations, papers, or links to external resources. 
The notion of packs have been widely adopted by scientists. At the time of writing, myExperiment had $337$ packs. Just like a workflow, using myExperiment a pack can be annotated and shared. 
 
In order to support complex forms of sharing, reuse and preservation, we have worked during the last year on incorporating the notion of research objects into the development version of myExperiment \footnote{http://alpha.myexperiment.org/packs/}. In addition to the basic aggregation supported by packs, alpha myExperiment provides the mechanisms for specifying metadata that describes the relationships between the resources within the aggregation. Moreover, the structure and the types of the resources that compose a pack are now inline with those that have been identifying thanks to the research object model. For example, a user is able to specify that a given file within a pack specifies the hypothesis, that another file specifies the workflow run obtained by enacting a given workflow, or that a given file states the conclusions drew by the scientists after analyzing the workflow run.
