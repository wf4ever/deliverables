\section{Research Object Manager}
\label{sec:manager}

The Research Object Manager provides a command line tool for creating, displaying and manipulating Research Objects. The RO Manager functionality is complementary to that provided by the ROSRS described in Section~\ref{sec:rosrs}. In particular, the RO Manager is primarily designed to support a user working with ROs in the host computer's local file system, with the intention being that the ROSRS and RO Manager can exchange ROs between them -- in part facilitated by the use of the shared RO vocabulary and model.

Past experience has suggested that lightweight, command line tools give users early access to functionality and provide an opportunity to gather additional feedback and requirements on that functionality. Command line tools can also be used with built in operating system functionalty as pipes and input/output redirection in order to quickly build prototype tool chains.

The RO Manager allows users to 
\begin{itemize}
\item Create local ROs;
\item Add resources to an RO;
\item Add annotations to an RO;
\item Read and write ROs to the RODL.
\end{itemize}

As with the ROSRS, the RO Manager uses the RO Model to structure and describe the objects it creates.

Further information describing the details of the RO Manager are contained in D1.4v1 (Reference Wf4Ever Implementation -- Phase I)~\cite{D1.4v1}.
