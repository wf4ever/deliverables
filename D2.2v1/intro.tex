\section{Introduction}

This deliverable describes aspects of Phase I of the design,
implementation and deployment of the Wf4Ever components that will
support workflow lifecycle management. The document should be read in
tandem with other Month 20 deliverables, in particular
D3.2v1~\cite{D3.2v1} and D4.2v1~\cite{D4.2v1} which adress
complementary aspects of the overall wf4ver architecture and
components.

According to the Description of Work, \emph{This prototype will
  include the following functionalities: an initial Research Object
  model, implemented by means of an ontology network, and basic
  management functions (storage and access), validation
  functionalities based on RO provenance, and definition of semantic
  overlays and workflow provenance matching techniques for
  abstraction.}. 

These requirements are addressed in the following way:

Sections~\ref{sec:model}, \ref{sec:primer} and \ref{sec:examples}
discuss the Research Object Model defined within Wf4Ever along with a
Primer document providing an introduction to that model and a
collection of example Research Objects. 

Sections~\ref{sec:rosrs} and \ref{sec:manager} describe the initial
Research Object Storage and Retrieval Service and Command Line
Manager. Both of these tools use the Research Object Model to
structure the objects that they produce and consume. The RO Model is
thus the ``glue'' that joins together the components and enables
interoperation. 

Section~\ref{sec:abstraction} discusses an initial model for workflow
abstraction, while Section~\ref{sec:decay} presents a characterisation
of workflow decay. 

Note that this document represents the results from Phase I of the
project -- as a result, some areas are not yet complete and we expect
updates, changes and extensions to be reported in Phase II of the
project, due for completion in M32. For example, we expect that RO models reported
here will be subject to change following further usage and experience,
both within and outside the project.